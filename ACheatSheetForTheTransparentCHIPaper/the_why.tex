% !TEX root = 1_Main.tex

%%% This is all first draft/getting ideas into place. -- Jan
\begin{block}{What is Transparency}
   \begin{quote}
       Having one\textquotesingle s actions open and accessible for external evaluation. Transparency pertains to researchers being honest about theoretical, methodological, and analytical decisions made throughout the research cycle.\footnotemark[1]
   \end{quote} 
   Framework for Open and Reproducible Research Training (\href{https://forrt.org/glossary/transparency/}{FORRT.org})  
  
  \blocksubtitle{Why be transparent?}
  Transparency is not a purely selfless act as it...
  \begin{itemize}
  \item ...helps readers and reviewers understand and judge your work.
  \item ...helps you stay on top of your work.
  \item ...prevents mistakes.
  \item ...helps you work faster.
  \item ...creates better research. 
  \item ...increases citation count and promotes reuse
  \end{itemize}  
  Being transparent benefits you as much as others~\footnotemark[2]. 
  %\cite{markowetz_five_2015}:
  
%   \begin{itemize}
%       \item Work aimed to be made transparent has to be kept orderly and readable. This is easier for everyone (yourself and others) to follow and understand. 
%       \item Well kept materials are more easily reused by others and yourself
%       \item Transparent materials and data allow reviewers to more easily follow your argument
%       \item Transparent data/research materials can be checked more easily for mistakes, making it easier to catch them before publication
%   \end{itemize}


  %\blocksubtitle{Is transparent research inherently better?}
  %\begin{itemize}
   %   \item Not necessarily, \textbf{but} 
      %\item going through pre-registration forces you to think about your study before you commit 
     % \item keeping track of your work prevents mistakes
     % \item making data and materials available increase citation counts
      
      
 %     (e.g., what is my dependent variable, how do I measure it). This can help you keep track of your works intended core idea and save you from uncomfortable surprises at a point where you can no longer change certain things (e.g., after data collection).
 % \end{itemize}
 
% \blocksubtitle{\faIcon{skull-crossbones}  What do we want to avoid? \faIcon{skull-crossbones}}
  
%   \begin{itemize}
%       \item Questionable Research Practices (QRP) e.g.,
%       \item HARKing (Hypothesizing After Result Known)
%       \item p-hacking 
%       \item cherry-picking
%       \item data dredging
%   \end{itemize}

 
%  \blocksubtitle{How to be Transparent without (much) Extra Work?}
%  \begin{itemize}
%    \item Tacking on open science practices at the end of a project %often feels like extra work
%    \item Integrate transparency from day one of your research (check the timeline on the back).
%  \end{itemize}
%   Open Science practices can seem like extra work, particularly if you when having to implement them into an existing project. However, when you are intending to be transparent at publication from the start, the required work can be (almost) seemingly integrated into your normal work. Subsequently it does not require much more time or effort during any single step of research.
  
 
  
  \blocksubtitle{What should I make transparent?}
  Treat transparency as the default, i.e., everything you produce during the project. However:\\
  \textbf{Your participants safety and rights come first!} \\
  Some data cannot be shared safely. Use data availability statement to report what can and can't be shared. Even if some things must be kept private, share   what materials you can (interview questions, analysis code, or other materials) with whom you can (reviewers, qualified experts, etc.). If you find yourself about to collect data that cannot be shared safely, also ask yourself if you really need this data in the first place.

  \blocksubtitle{Isn't being transparent more work?}
  
  Yes. Especially if you have to make a lot of things transparent at once on short notice (e.g., at the request of reviewers) and suddenly have to deal with missing participant consent, DRM issues, or missing items.
  
  However: If you are transparent from the beginning, the additional work per step is minimal, and -- as outlined above -- the effort can actually safe you significant time and heartache down the line when having to make sense of/wanting to reuse older materials.

\end{block}

\footnotetext[1]{Parsons, S., Azevedo, F., Elsherif, M. M., Guay, S., Shahim, O. N., Govaart, G. H., ... \& Aczel, B. (2022). A Community-Sourced Glossary of Open Scholarship Terms. Nature human behaviour, 6(3), 312-318. https://doi.org/10.1038/s41562-021-01269-4}
\footnotetext[2]{Markowetz, F. (2015). Five selfish reasons to work reproducibly. Genome biology, 16(1), 1-4.}